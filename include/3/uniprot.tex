1. Quais bancos de dados ou conjunto de dados podem ser acessados pelo UniProt?



2. Diferencie o UniProtKb/trEMBL e o UniProtKb/Swiss-Prot.



3. Em que consiste a anotação manual?



4. Quantas entradas estão presentes no UniProtKB relacionadas com bomba de efluxo (“efflux pump”)?

a) Quantas destas estão presentes no Swiss-Prot e quantas estão no trEMBL? 
 
b) Quantas entradas relacionadas com bomba de efluxo presentes no Swiss-Prot pertencem ao organismo Bacillus subtilis (BACSU)?



5. Procurar a entrada Q4R9Z3 no UniProtKB.

a) Em qual seção do UniProtKB encontra-se esta entrada?

b) A qual organismo pertence?

c) A qual família essa proteína supostamente pertenceria?

d) Essa entrada poderia ser integrada no Swiss-Prot? Justifique sua resposta.



6- Procurar a proteína “toll-like receptor 4” de humanos no UniProtKB/Swiss-Prot.

a) quais outros nomes podem designar esta proteína?

b) Quais domínios e repetições estão associados a esta proteína? Cite as referências cruzadas (bancos de dados) que suportam sua resposta.

c) Qual a função desta proteína?
d) Existem sequências alternativas para esta proteína? Em caso positivo, listar os identificadores.

7-Encontrar uma entrada no Swiss-Prot que seja fosforilada (dica: usar uma KW) e que tenha domínio transmembrana (dica: usar uma KW) no organismo Staphylococcus aureus strain N315.
    
a) Indique o número de acessos de proteínas. 

b) Qual o nome e sinônimos dessas proteínas e suas funções?

c) Fornecer o número EC, caso sejam enzimas.

d) Qual o nome dos genes?

e)  Qual o código usado para designar a espécie S. aureus strain N315 no Swiss-Prot?

f)  Qual tipo de modificação essas proteínas sofrem? Em qual resíduo? Indique a posição na sequência de cada.

g) Caso as proteínas tenham ligação à metal(is), citar o(s) metal(is) e em qual(is) posição(ões)?

h) Existem estruturas 3D para essas proteínas?
