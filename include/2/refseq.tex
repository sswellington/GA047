\section{Banco de dados de sequência de Proteínas}

1. Quais os principais bancos de dados de sequências de proteínas? Quais outros bancos os compõem?


2. O que é o RefSeq Database e quais suas principais características?


3. Quais as diferenças entre o GenBank e o RefSeq?


4. Qual a descrição dos seguintes ``status'' de uma entrada no REfSeq: model, predicted, inferred, provisional, reviewed, validated, WGS.


5. Fazendo uma busca no All databases, em quantos bancos de dados do NCBI podemos encontrar resultados para a proteína “myosin”? Quantas estruturas 3D existem relacionadas com esta proteína? 


6. Buscar uma entrada para a proteína “calmodulin-1” de humano no banco de dados Protein. 



a) Selecionar aquelas que estão presentes no RefSeq. Existem isoformas para essa proteína? Em caso afirmativo, forneça o número de acesso para as entradas no RefSeq.


b) Escolha uma das isoformas e responda:. 

c)  Como é o formato de uma sequência FASTA? Forneça a sequência no formato FASTA desta entrada.  


d)  Qual o “status” desta entrada e o que esse status significa?

    
e)  Qual a função dessa proteína, caso esteja descrito
    
    
f) Existe algum domínio conservado associado a esta proteína? Em caso afirmativo, citar a família e o(s) banco(s) de dados que corroboram a informação.


g) Utilizando a ferramenta BLASTp encontre sequências com 100\% de identidade no banco de dados UniProtKb/Swiss-Prot para o mesmo organismo. Qual o número de acesso no Swiss-Prot?



7. Buscar proteínas homólogas á entrada angiotensin converting enzyme 2 de humano no banco ``nr''.


a) Filtrar as entradas entre 90 e 100\% de identidade da sequência.


b) Quais organismos estão representados nesse conjunto selecionado?


c) Qual o resultado do alinhamento com a sequência homologa de Macaca mulatta?
 